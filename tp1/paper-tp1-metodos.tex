\documentclass[10pt,a4paper]{article}
\usepackage[spanish]{babel}%corta palabras en espa�ol
\usepackage[latin1]{inputenc}%escribir con acentos, �
\begin{document}
$$ \mbox{\bf Universidad de Buenos Aires} $$
$$ \mbox{Facultad de Ciencias Ex�ctas y Naturales} $$
$$ \mbox{(FCEyN)} $$
$$ \mbox{Departamento de Computaci�n} $$

{\large Trabajo Pr�ctico N�mero 1: Esp�a por error (num�rico)}

\medskip


{Gonz�lez Sergio (gonzalezsergio2003@yahoo.com.ar)}

{Gonz�lez Emiliano (XJesse\_JamesX@hotmail.com)}

{Mariano (eltrencitomasverde@gmail.com)}

\medskip
Pal�bras clave:

Res�men:
\medskip

\pagebreak
{\bf Introducci�n:}

El an�lisis num�rico basicamente se encarga de analizar, describ�r y cre�r algoritmos num�ricos que permiten resolver problemas matem�ticos. Estos algoritmos generalmente nos permiten obten�r resultados aproximados, ya que contienen un n�mero fin�to de pasos. El uso del an�lisis num�rico toma gran importancia con el �so de las computadoras y el poder de calculo que ellas tienen. Por este medio, es posible resolver problemas mas complejos.
Pero el uso de computadoras para hacer calculos complejos trae un problema con sigo, y surge el conc�pto de error. Este conc�pto nace gr�cias a que las computadoras trabajan con un rango finito de n�meros, y ademas cada uno de estos est� representado de una forma tambien fin�ta.
Los err�res est�n divididos en tr�s tipos: Err�res en los datos de entrada, err�res de redondeo y err�res de truncamiento. Los err�res en los datos de entrada no estan causados por el algoritmo que resuelve el problema, sino por valores que inician el algoritmo, generalmente estos valores se refieren a mediciones o magnitudes f�sicas. Los err�res de redondeo surgen cuando se utilizan operaciones que tienen una representacion numerica finita, esto significa que tienen una presici�n limitada con respecto al resultado que devuelven. Y los err�res de truncamiento estan relacionados con el algoritmo en si, esto quiere decir que dependen de la forma en que se resuelve el problema, en algunos casos el error de truncamiento se puede disminuir modificando o refinando el algorimo, esto generalmente implica aumentar la cantidad de cuentas a hacer y por lo tanto aumentar el error de redondeo y el tiempo para resolver el problema.
Por estos motivos se trata de buscar un algoritmo que lleve a la mejor soluci�n posible de cada problema, una de las tareas del analisis numerico. De esta forma se pueden definir muchos algoritmos que lleguen a la solucion, pero se seleccionar� el que mejor aproxime al problema, por lo tanto, el que mejor utilice las operaciones respecto de sus errores. En este caso surge la noci�n de estabilidad num�rica. La estabilidad numerica define cuan buena sera la solucion de nuestro problema usando metodos aproximados.
\end{document}
